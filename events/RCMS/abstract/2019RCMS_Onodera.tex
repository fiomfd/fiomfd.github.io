%
% TeX A4サイズのサンプル
% 
\documentclass[a4paper,reqno,12pt]{article}
%\usepackage{latexsym}
\usepackage[dvips]{graphicx}
\usepackage{color}
%\usepackage{amsintx}
%\usepackage{amsxtra}
\usepackage{ascmac}
\usepackage{amsmath}
\usepackage{amstext}
\usepackage{amsbsy}
\usepackage{amsopn}
\usepackage{upref}
\usepackage{amsthm}
\usepackage{amsfonts}
\usepackage{amssymb}
\usepackage{mathrsfs}
\usepackage{times}
\usepackage{bm}
\allowdisplaybreaks
%%%%%%
%%%%%% 長さなどの設定 
%%%%%%
%\setlength{\columnsep}{20mm}
\setlength{\textheight}{230mm}
\setlength{\textwidth}{160mm}
\setlength{\oddsidemargin}{0mm}
\setlength{\evensidemargin}{0mm}
\setlength{\topmargin}{-15mm}
%\setlength{\footskip}{10mm}
%\setlength{\baselineskip}{14pt}
%\setlength{\parindent}{24pt}
% 頁番号が不要ならば次の行の % を消す
%\pagestyle{empty}
%\numberwithin{equation}{section}
%%%%%%
%%%%%% 定理などの環境
%%%%%%
\theoremstyle{plain}
 \newtheorem{theorem}{Theorem}
 \newtheorem{corolally}[theorem]{Corollary}
 \newtheorem{proposition}[theorem]{Proposition}
 \newtheorem{lemma}[theorem]{Lemma}
\theoremstyle{definition}
 \newtheorem{definition}[theorem]{Definition}
\theoremstyle{remark}
 \newtheorem{remark}[theorem]{Remark}
%%%%%%
%%%%%% 自前のコマンドの定義
%%%%%%
\newcommand{\sgn}{\operatorname{sgn}}
\newcommand{\supp}{\operatorname{supp}}
\newcommand{\re}{\operatorname{Re}}
\newcommand{\im}{\operatorname{Im}}
\newcommand{\rot}{\operatorname{rot}}
\newcommand{\grad}{\operatorname{grad}}
\newcommand{\divergence}{\operatorname{div}}
%%%%%%
%%%%%% Begining Document
%%%%%%
\begin{document}
%%%%%%
%%%%%% 
%%%%%%
\begin{center}
\begin{Large}
閉リーマン面上の閉曲線流がみたす4階分散型偏微分方程式の初期値問題
\vspace{12pt}
\\
小野寺 栄治(高知大学)
\end{Large}
\end{center}
%
%
\par
%
%
閉リーマン面上の閉曲線流がみたすある空間1次元4階非線型分散型偏微分方程式に
対する初期値問題の解の存在問題を考察する。この方程式は、1次元古典スピン系の
連続体近似モデルや渦糸の3次元運動のモデルに関連して導出された実2次元球面上
の曲線流に対するある偏微分方程式系を幾何学的に一般化したものである。この問題
では、解が曲がった多様体に値を取るため、偏微分方程式に含まれる低階項の構造が
多様体に対する設定と密接に関連する。また、閉曲線流を考えるので、開曲線流を考
える場合と比べて分散型偏微分方程式の解に対するある種の平滑化効果を使えないと
いう意味で、初期値問題が解けるための方程式の構造に余裕がない状況にある。本研
究では、閉リーマン面の断面曲率(ガウス曲率)が一定ならば時間局所解の存在と一
意性が従うことがわかった。講演では、関連研究、証明の方針と曲率に対する仮定の
意味、問題設定を少し変えた場合などの考察、について時間の許す範囲で報告させて
いただきたい。
%
%
%%%%%%
%%%%%% End
%%%%%%
\end{document}
%%%%%%
%%%%%%
%%%%%%
