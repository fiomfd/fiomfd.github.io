%
% TeX A4サイズのサンプル
% 
\documentclass[a4paper,reqno,12pt]{article}
%\usepackage{latexsym}
\usepackage[dvips]{graphicx}
\usepackage{color}
%\usepackage{amsintx}
%\usepackage{amsxtra}
\usepackage{ascmac}
\usepackage{amsmath}
\usepackage{amstext}
\usepackage{amsbsy}
\usepackage{amsopn}
\usepackage{upref}
\usepackage{amsthm}
\usepackage{amsfonts}
\usepackage{amssymb}
\usepackage{mathrsfs}
\usepackage{times}
\usepackage{bm}
\allowdisplaybreaks
%%%%%%
%%%%%% 長さなどの設定 
%%%%%%
%\setlength{\columnsep}{20mm}
\setlength{\textheight}{230mm}
\setlength{\textwidth}{160mm}
\setlength{\oddsidemargin}{0mm}
\setlength{\evensidemargin}{0mm}
\setlength{\topmargin}{-15mm}
%\setlength{\footskip}{10mm}
%\setlength{\baselineskip}{14pt}
%\setlength{\parindent}{24pt}
% 頁番号が不要ならば次の行の % を消す
%\pagestyle{empty}
%\numberwithin{equation}{section}
%%%%%%
%%%%%% 定理などの環境
%%%%%%
\theoremstyle{plain}
 \newtheorem{theorem}{Theorem}
 \newtheorem{corolally}[theorem]{Corollary}
 \newtheorem{proposition}[theorem]{Proposition}
 \newtheorem{lemma}[theorem]{Lemma}
\theoremstyle{definition}
 \newtheorem{definition}[theorem]{Definition}
\theoremstyle{remark}
 \newtheorem{remark}[theorem]{Remark}
%%%%%%
%%%%%% 自前のコマンドの定義
%%%%%%
\newcommand{\sgn}{\operatorname{sgn}}
\newcommand{\supp}{\operatorname{supp}}
\newcommand{\re}{\operatorname{Re}}
\newcommand{\im}{\operatorname{Im}}
\newcommand{\rot}{\operatorname{rot}}
\newcommand{\grad}{\operatorname{grad}}
\newcommand{\divergence}{\operatorname{div}}
%%%%%%
%%%%%% Begining Document
%%%%%%
\begin{document}
%%%%%%
%%%%%% 
%%%%%%
\begin{center}
\begin{Large}
Recent development of solvable models for Aharonov-Bohm type magnetic fields
\vspace{12pt}
\\
{\sc Takuya Mine} 
(Kyoto Institute of Technology)
\end{Large}
\end{center}
%
%
\vspace{12pt}
\par
%
%
The Schr\"odinger operator with a delta-like magnetic
field (the Aharonov-Bohm magnetic field) in the Euclidean
plane is known as an example of solvable models,
in the sense we can calculate the incoming plane wave
and the scattering amplitude, explicitly. In this talk, we will
introduce some examples of solvable models related with
the Aharonov-Bohm magnetic field, and give explicit formulas
for the incoming plane wave and the scattering amplitude.
We will also show some graphical results created by the
resulting formulas.
%
%
%
%
%%%%%%
%%%%%% End
%%%%%%
\end{document}
%%%%%%
%%%%%%
%%%%%%
