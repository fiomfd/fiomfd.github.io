%
% TeX A4サイズのサンプル
% 
\documentclass[a4paper,reqno,12pt]{article}
%\usepackage{latexsym}
\usepackage[dvips]{graphicx}
\usepackage{color}
%\usepackage{amsintx}
%\usepackage{amsxtra}
\usepackage{ascmac}
\usepackage{amsmath}
\usepackage{amstext}
\usepackage{amsbsy}
\usepackage{amsopn}
\usepackage{upref}
\usepackage{amsthm}
\usepackage{amsfonts}
\usepackage{amssymb}
\usepackage{mathrsfs}
\usepackage{times}
\usepackage{bm}
\allowdisplaybreaks
%%%%%%
%%%%%% 長さなどの設定 
%%%%%%
%\setlength{\columnsep}{20mm}
\setlength{\textheight}{230mm}
\setlength{\textwidth}{160mm}
\setlength{\oddsidemargin}{0mm}
\setlength{\evensidemargin}{0mm}
\setlength{\topmargin}{-15mm}
%\setlength{\footskip}{10mm}
%\setlength{\baselineskip}{14pt}
%\setlength{\parindent}{24pt}
% 頁番号が不要ならば次の行の % を消す
\pagestyle{empty}
%\numberwithin{equation}{section}
%%%%%%
%%%%%% 定理などの環境
%%%%%%
\theoremstyle{plain}
 \newtheorem{theorem}{Theorem}
 \newtheorem{corolally}[theorem]{Corollary}
 \newtheorem{proposition}[theorem]{Proposition}
 \newtheorem{lemma}[theorem]{Lemma}
\theoremstyle{definition}
 \newtheorem{definition}[theorem]{Definition}
\theoremstyle{remark}
 \newtheorem{remark}[theorem]{Remark}
%%%%%%
%%%%%% 自前のコマンドの定義
%%%%%%
\newcommand{\sgn}{\operatorname{sgn}}
\newcommand{\supp}{\operatorname{supp}}
\newcommand{\re}{\operatorname{Re}}
\newcommand{\im}{\operatorname{Im}}
\newcommand{\rot}{\operatorname{rot}}
\newcommand{\grad}{\operatorname{grad}}
\newcommand{\divergence}{\operatorname{div}}
%%%%%%
%%%%%% Begining Document
%%%%%%
\begin{document}
%%%%%%
%%%%%% 
%%%%%%
\begin{center}
\begin{Large}
Riemannian submersion and Maslov quantization condition
\vspace{12pt}
\\
Kenro Furutani (Tokyo University of Science)
\end{Large}
\end{center}
%
%
\par
We consider a relation of spectra of Laplacians on the total space and the base space
of a Riemannian submersion. If the Riemannian submersion  commutes with the Laplacians, then the spectrum of the base space are a part of that of the total space and in this case the fibers of the Riemann submersion must be minimal submanifolds, and vise versa. We are interesting in this talk, what is happening when the  Riemannian  submersion is general. So we talk, 

\begin{itemize}
\item[(1)] 
Eigenvalue Theorem by Weinstein together with  a review of Maslov quantization condition
\item[(2)] 
Behavior of Lagrangian submanifolds under submersion
\item[(3)] 
Some examples
\end{itemize}

This is a report of partial results with M. Tamura.
%
%
%
%
%%%%%%
%%%%%% End
%%%%%%
\end{document}
%%%%%%
%%%%%%
%%%%%%
