%
% TeX A4サイズのサンプル
% 
\documentclass[a4paper,reqno,12pt]{article}
%\usepackage{latexsym}
\usepackage[dvips]{graphicx}
\usepackage{color}
%\usepackage{amsintx}
%\usepackage{amsxtra}
\usepackage{ascmac}
\usepackage{amsmath}
\usepackage{amstext}
\usepackage{amsbsy}
\usepackage{amsopn}
\usepackage{upref}
\usepackage{amsthm}
\usepackage{amsfonts}
\usepackage{amssymb}
\usepackage{mathrsfs}
\usepackage{times}
\usepackage{bm}
\allowdisplaybreaks
%%%%%%
%%%%%% 長さなどの設定 
%%%%%%
%\setlength{\columnsep}{20mm}
\setlength{\textheight}{230mm}
\setlength{\textwidth}{160mm}
\setlength{\oddsidemargin}{0mm}
\setlength{\evensidemargin}{0mm}
\setlength{\topmargin}{-15mm}
%\setlength{\footskip}{10mm}
%\setlength{\baselineskip}{14pt}
%\setlength{\parindent}{24pt}
% 頁番号が不要ならば次の行の % を消す
%\pagestyle{empty}
%\numberwithin{equation}{section}
%%%%%%
%%%%%% 定理などの環境
%%%%%%
\theoremstyle{plain}
 \newtheorem{theorem}{Theorem}
 \newtheorem{corolally}[theorem]{Corollary}
 \newtheorem{proposition}[theorem]{Proposition}
 \newtheorem{lemma}[theorem]{Lemma}
\theoremstyle{definition}
 \newtheorem{definition}[theorem]{Definition}
\theoremstyle{remark}
 \newtheorem{remark}[theorem]{Remark}
%%%%%%
%%%%%% 自前のコマンドの定義
%%%%%%
\newcommand{\sgn}{\operatorname{sgn}}
\newcommand{\supp}{\operatorname{supp}}
\newcommand{\re}{\operatorname{Re}}
\newcommand{\im}{\operatorname{Im}}
\newcommand{\rot}{\operatorname{rot}}
\newcommand{\grad}{\operatorname{grad}}
\newcommand{\divergence}{\operatorname{div}}
%%%%%%
%%%%%% Begining Document
%%%%%%
\begin{document}
%%%%%%
%%%%%% 
%%%%%%
\begin{center}
\begin{Large}
Spectral theory of Neumann--Poincar\'e operators and its 
applications
\vspace{12pt}
\\
{\sc Yoshihisa Miyanishi} (Osaka University)
\end{Large}
\end{center}
%
%
\vspace{12pt}
\par
%
%
The Neumann--Poincar\'e operator (abbreviated by NP) is a boundary 
integral operator naturally arising when solving classical boundary 
value problems using layer potentials. If the boundary of the domain, 
on which the NP operator is defined, is $C^{1, \alpha}$ smooth, then 
the NP operator is compact. Thus, the Fredholm integral equation, 
which appears when solving Dirichlet or Neumann problems, can be 
solved using the Fredholm index theory. If the domain has corners, the 
NP operator is not a compact operator any more, but a singular 
integral operator. The solvability of the corresponding integral 
equation was established by Verchota.
 Regarding spectral properties of the NP operator, the spectrum 
consists of eigenvalues converging to $0$ for $C^{1, \alpha}$ smooth 
boundaries. The NP operator, not self-adjoint, generally, in $L^2$; 
can be however realized as a self-adjoint operator in the 
$H^{-1/2}$-space, provided a new inner product is introduced, and 
therefore the NP spectrum is real and may consist of a continuous 
spectrum and a discrete spectrum (and possibly limit points of the 
discrete spectrum). If the domain has corners, the corresponding NP 
operator, in fact, possess a continuous spectrum (as well as 
eigenvalues).
\par
 Our main purpose here is to introduce the spectral properties of NP 
operators. Then we discuss inverse problems and plasmon eigenvalues as 
applications.

%
%
%
%
%%%%%%
%%%%%% End
%%%%%%
\end{document}
%%%%%%
%%%%%%
%%%%%%
