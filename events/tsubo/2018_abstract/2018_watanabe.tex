%
% TeX A4サイズのサンプル
% 
\documentclass[a4paper,reqno,11pt]{article}
%\usepackage{latexsym}
\usepackage[dvips]{graphicx}
\usepackage{color}
%\usepackage{amsintx}
%\usepackage{amsxtra}
\usepackage{ascmac}
\usepackage{amsmath}
\usepackage{amstext}
\usepackage{amsbsy}
\usepackage{amsopn}
\usepackage{upref}
\usepackage{amsthm}
\usepackage{amsfonts}
\usepackage{amssymb}
\usepackage{mathrsfs}
\usepackage{times}
\usepackage{bm}
\allowdisplaybreaks
%%%%%%
%%%%%% 長さなどの設定 
%%%%%%
%\setlength{\columnsep}{20mm}
\setlength{\textheight}{230mm}
\setlength{\textwidth}{160mm}
\setlength{\oddsidemargin}{0mm}
\setlength{\evensidemargin}{0mm}
\setlength{\topmargin}{-15mm}
%\setlength{\footskip}{10mm}
%\setlength{\baselineskip}{14pt}
%\setlength{\parindent}{24pt}
% 頁番号が不要ならば次の行の % を消す
%\pagestyle{empty}
%\numberwithin{equation}{section}
%%%%%%
%%%%%% 定理などの環境
%%%%%%
\theoremstyle{plain}
 \newtheorem{theorem}{Theorem}
 \newtheorem{corolally}[theorem]{Corollary}
 \newtheorem{proposition}[theorem]{Proposition}
 \newtheorem{lemma}[theorem]{Lemma}
\theoremstyle{definition}
 \newtheorem{definition}[theorem]{Definition}
\theoremstyle{remark}
 \newtheorem{remark}[theorem]{Remark}
%%%%%%
%%%%%% 自前のコマンドの定義
%%%%%%
\newcommand{\sgn}{\operatorname{sgn}}
\newcommand{\supp}{\operatorname{supp}}
\newcommand{\re}{\operatorname{Re}}
\newcommand{\im}{\operatorname{Im}}
\newcommand{\rot}{\operatorname{rot}}
\newcommand{\grad}{\operatorname{grad}}
\newcommand{\divergence}{\operatorname{div}}
%%%%%%
%%%%%% Begining Document
%%%%%%
\begin{document}
%%%%%%
%%%%%% 
%%%%%%
\begin{center}
\begin{Large}
{\bf Semiclassical distribution of resonances associated 
\\
with an energy-level crossing}
\vspace{12pt}
\\
{\sc Takuya Watanabe} (Ritsumeikan University)
\end{Large}
\end{center}
%
%
\par
%
%
We would like to give reccent progress of our works, in which we study the semiclassical distribution of quantum resonances for a $2\times2$ matrix-valued Schroedinger operator with an energy-level crossing. Concerning a $2\times2$ matrix-valued Schroedinger operator, there are not so many results to investigate resonances except the case of the tonneling effect (i.e. the imaginary part of resonance is exponrntially small) even if energy-levels do not cross each other.  
%
%
\par
%
%
In this talk, we first introduce the result of the semiclassical distribution of quantum resonances near crossing level and second give the work in progress about the resonances above crossing level. We would like to compare them and to explain each difficulty.
These works are joint works with S.~Fujiie (Ritsumeikan) and A.~Martinez (Bologna).  
%
%
%
%
%%%%%%
%%%%%% End
%%%%%%
\end{document}
%%%%%%
%%%%%%
%%%%%%
